%Begin ----  Para que funcione bien el TOC en PDF
\cleardoublepage
\phantomsection
\addcontentsline{toc}{chapter}{Agradecimientos}

\vspace*{2.5cm}


\begin{quotation}
\noindent \begin{center}
\textbf{\emph{\Large Agradecimientos:}}\textbf{\emph{\large }}\\
\textbf{\emph{\large }}\\
\textbf{\emph{\large }}\\
\textbf{\emph{\large }}
\par\end{center}{\large \par}
\end{quotation}

\begin{onehalfspace}

Una carrera universitaria es una etapa de la vida, tal vez aquella en la que se tiene la máxima libertad para hacerse a uno mismo, y en ello influyen todas las personas que nos rodean. He tenido la suerte de estar con los mejores, y les debo mi eterno agradecimiento a todos.

\begin{quote}
	A Papá y Mamá, por la vida y el cariño. \\
	A Laura, por ser la mejor hermana. \\
	A Daniel y Francisco, por las risas. \\
	A Paco y José Antonio, por ir un paso por delante. \\
	A Manolo y Jesús, por la Alianza. \\
	A la Peñita de Milán, por el Erasmus. \\
	A los Milestones, por la música. \\
	Y a Salva, por meterme en el mundo de la informática.
\end{quote}

También quiero expresar mi agradecimiento a todos los profesores de la Universidad de Granada por su inestimable esfuerzo por formarnos, así como a todas las personas implicadas en este proyecto: D. Andrés Roldán Aranda y Dª María Isabel García Arenas, mis tutores, Mikel Aguayo Fernández, ingeniero del hardware, y el organista Juan Rodríguez Ruiz, nuestro más valioso colaborador.

\end{onehalfspace}

%\bigskip
%Todos son igualmente partícipes de este trabajo.

\clearpage{\pagestyle{empty}\cleardoublepage}%
