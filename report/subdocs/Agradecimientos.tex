%Begin ----  Para que funcione bien el TOC en PDF
\cleardoublepage
\phantomsection
\addcontentsline{toc}{chapter}{Agradecimientos}

\vspace*{2.5cm}


\begin{quotation}
\noindent \begin{center}
\textbf{\emph{\Large Agradecimientos:}}\textbf{\emph{\large }}\\
\textbf{\emph{\large }}\\
\textbf{\emph{\large }}\\
\textbf{\emph{\large }}
%Todos aquellos que no tuvieron la oportunidad.
\par\end{center}{\large \par}
\end{quotation}

\begin{onehalfspace}

En primer lugar, quisiera expresar todo mi agradecimiento a mi familia. Gracias a mis padres por el apoyo constante y la confianza depositada, por haber estado siempre a mi lado y por todo el esfuerzo que han dedicado a formarme. Gracias por comprenderme en todo momento y sacar lo mejor de mí en los momentos de flaqueza. Personalmente, gracias a mi padre por inculcarme siempre valores de esfuerzo y superación. A mi madre, gracias por conocerme como ninguna otra persona y saber sacarme siempre una sonrisa. Sin vosotros no habría llegado hasta aquí.


No quisiera olvidarme del resto de mi familia. Gracias a todos ellos por su afecto, especialmente a mis abuelos por haberme tenido siempre presente.

Mostrar también mis agradecimientos al Departamento de Electrónica y Tecnología de los Computadores de la Universidad de Granada, y muy especialmente a mi tutor, Andrés María Roldán Aranda. Gracias por tu constante dedicación, atención y supervisión durante todo este tiempo. Por tu trato cercano y amable, y por todo el conocimiento transmitido.



\end{onehalfspace}

\bigskip

Todos son igualmente partícipes de este trabajo.

\clearpage{\pagestyle{empty}\cleardoublepage}%
