\chapter{Introducción y objetivos}
\label{cap:capitulo_1}
\pagenumbering{arabic}

\begin{quote}
	\small \flushright ''\textit{Una partitura también es un lenguaje.}'' \\
	--- Profesor Buenaventura Clares Rodríguez (2013).
\end{quote}

\section{Contexto}

Llegado el momento de cursar la asignatura ''Proyectos Informáticos'' de la titulación, iniciamos los preparativos para desarrollar un proyecto propuesto por el Departamento de Electrónica y Tecnología de Computadores de la Universidad de Granada.

Para plantearlo, nos hemos fijado en numerosas iglesias de Granada, que incorporan órganos de tubos, instrumentos muy complejos, muchos de ellos formando parte del inmobiliario, y merecedores de un gran reconocimiento por la artesanía y la calidad de su construcción. Lamentablemente, muchos de ellos están en un estado de abandono, debido principalmente a que si no se tocan regularmente, se deterioran y no se reparan, y al no hacerlo, no se pueden tocar, cayendo en un círculo vicioso.

Además, creemos interesante la idea de que se pueda hacer sonar un órgano aunque no haya organista, dando la posibilidad tanto de acompañar celebraciones litúrgicas como de tener música de fondo durante el horario de visitas.

\newpage

\section{Objetivos}

Nuestra meta es, por tanto, ingeniar un sistema automático capaz de interpretar piezas musicales para órgano, lo que incluye la creación de un sistema mecánico capaz de suplantar las pulsaciones del artista y el desarrollo del \textit{software} necesario para que el sistema reproduzca partituras electrónicas en formato MIDI.

Este proyecto atenderá al objetivo global pero se centrará en la parte \textit{software}, ya que el \textit{hardware} requiere competencias de Ingeniería Electrónica e Industrial, y será objeto del proyecto de D. Mikel Aguayo Fernández.

Para abordarlo, hemos contado con la colaboración de Juan Rodríguez Ruiz, responsable del órgano de la Parroquia de la Encarnación de Santa Fe, que nos ha dado acceso tanto al instrumento como a asombrosos datos sobre su historia y su construcción.

\smallskip

\begin{figure}[H]
\noindent \begin{centering}
\includegraphics[width=\linewidth]{capitulo1/figura11}
\par\end{centering}
\smallskip
\caption[Parroquia de la Encarnación de Santa Fe.]{\label{fig:figura11} Parroquia de la Encarnación de Santa Fe. \cite{iglesias_granada}}
\end{figure}

\newpage

\section{Contenido y estructura capitular}

Una vez planteado el problema, utilizaremos el \textbf{modelo de desarrollo en cascada} para continuar el resto del proyecto.

El modelo en cascada ordena rigurosamente las etapas del proceso, de forma que cada tarea se inicia cuando su precedente finaliza, tal como se expone en la siguiente ilustración:

\smallskip

\begin{figure}[H]
	\noindent \begin{centering}
		\includegraphics[width=\linewidth/2]{capitulo1/figura12}
		\par\end{centering}
	\smallskip
	\caption[Modelo de desarrollo en cascada.]{\label{fig:figura12} Modelo de desarrollo en cascada. \cite{wiki_cascada}}
\end{figure} 

\smallskip

Basándonos en la figura anterior, definiremos el resto de capítulos de esta memoria.

\begin{itemize}

\item \textbf{Capítulo 2:} En este capítulo especificaremos los requisitos que supone el diseño del sistema para alcanzar nuestro objetivo, así como indicar las fases del proyecto.

\item \textbf{Capítulo 3:} Vamos a analizar todos los elementos con los que vamos a interactuar, desde el teclado del órgano hasta el computador sobre el que funcionará nuestro \textit{software}. 

\item \textbf{Capítulo 4:} Plantearemos el diseño de la solución, haciendo Ingeniería el \textit{software}, que cumpla de la mejor manera posible los requerimientos del \ref{cap:capitulo_2}.

\item \textbf{Capítulo 5:} En esta parte explicaremos cómo hemos implementado la solución diseñada, y cómo hemos enfrentado los principales problemas que han surgido.

\item \textbf{Capítulo 6:} Vamos a validar el sistema poniéndolo en funcionamiento y probando que cumple los requisitos propuestos.

\item \textbf{Capítulo 7:} Expondremos la conclusión y las posibilidades que nos brinda el proyecto de cara al futuro.
  
\end{itemize}

\newpage
\clearpage{\pagestyle{empty}\cleardoublepage}
