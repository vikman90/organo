\chapter{Introducción.}
\label{cap:capitulo_1}
\pagenumbering{arabic}%

\section{Contexto}

Llegado el momento de cursar la asignatura Proyecto Fin de Carrera correspondiente a
la titulación de Ingeniería de Telecomunicación, se pretende realizar un trabajo que esté
dentro de todos los ámbitos presentados en la Normativa de Realización de Proyectos Fin
de Carrera de la Universidad de Granada, y en el que se desarrolle una solución real a un
problema que puede encontrarse en el ejercicio de la profesión.
Esta solución consistirá en el análisis, diseño y creación de una \acrshort{PCB} que permita un control digital de la temperatura impuesta por un equipo termo-resistivo sobre un horno. 
El horno se encuentra en el laboratorio de proyectos de la Facultad de Ciencias donde dispondremos de un conjunto de equipos eléctricos y mecánicos para poder estudiar el horno, diseñar la \acrshort{PCB}  y testearla. 

\smallskip
\begin{figure}[H]%here
\noindent \begin{centering}
\includegraphics[scale=0.6]{capitulo1/horno}
\par\end{centering}
\smallskip
\caption{\label{fig:horno} Horno Heraeus.}
\end{figure} 


Dicho horno funciona mediante el calentamiento de dos resistencias óhmicas que, usando la energía eléctrica, se calientan por efecto Joule y ceden energía en forma de calor a la carga.



\section{Contenido y estructura capitular}

Una vez se ha introducido el tema principal de este proyecto, consideramos el diagrama de flujo de las fases de trabajo que llevaremos a cabo, figura \ref{fig:Flow_Chart_Project}.

\smallskip
\begin{figure}[H]%here
\noindent \begin{centering}
\includegraphics[scale=1]{capitulo1/Flow_Chart_Project}
\par\end{centering}
\caption{\label{fig:Flow_Chart_Project} Flujo de fases de trabajo.}
\end{figure}
\smallskip

Basándonos en dicho diagrama, se numerarán los diferentes capítulos de esta memoria. 

Cada capítulo constará de una breve descripción de su contenido.


\begin{itemize}

\item \textbf{Capítulo 2:} En este capítulo se detallan todos los requisitos y especificaciones \textit{hardware} y \textit{software} del sistema. Además se plantea la metodología de trabajo y se enumeran las fases del proyecto. 

\item \textbf{Capítulo 3:} Analizamos las necesidades del sistema y como podemos cubrirlas mediante los equipos electrónicos y mecánicos disponibles. Estudiamos en detalle el funcionamiento y composición del horno .  

\item \textbf{Capítulo 4:}  Se expondrán los procedimientos de diseño hardware e imple-
mentación software la \acrshort{PCB} que controlará nuestro horno. En esta sección se analizaran las soluciones tomadas para cumplir los requisitos propuestos en el capítulo \ref{cap:capitulo_2}.

\item \textbf{Capítulo 5:} Se explicarán los procedimientos realizados para la fabricación de la \acrshort{PCB} y la implementacion \textit{software} donde se explicarán las librerías usadas y los métodos realizados para el funcionamiento de nuestro sistema.

\item \textbf{Capítulo 6:} Validaremos el sistema al completo mediante la puesta en funcionamiento de nuestra \acrshort{PCB} y el analizaremos los resultados obtenidos.

\item \textbf{Capítulo 7:} Breve conclusión y mención a las líneas futuras asociadas al proyecto.   
  
\end{itemize}

\newpage
\clearpage{\pagestyle{empty}\cleardoublepage}
