\chapter{Conclusión y líneas futuras}
\label{cap:capitulo7}

\begin{quote}
	\begin{flushright}
		\small ''\textit{Pensad que lo malo ya ha pasado, y lo bueno está por llegar.}'' \\
		--- Profesor Juan Manuel López Soler (2013).
	\end{flushright}
\end{quote}

\newpage

\section{Conclusión}

Es curioso darse cuenta de cómo las cosas llegan cuando menos te lo esperas. Y así ocurrió, en mi primer año de universidad (curso 2008-2009), durante una clase de prácticas de Fundamentos Tecnológicos de los Computadores, cuando el profesor D. Andrés Roldán me propuso un proyecto basado en su idea original de que una máquina tocara el órgano. Un proyecto que no podría encajar mejor con mis dos mayores pasiones: la Informática y la Música, ya que estudié Piano en el conservatorio y formé un grupo musical hace algunos años.

Siempre resulta emocionante partir de una idea y un papel en blanco, y ser consciente de cómo evoluciona hasta hacerse realidad. Por fortuna, esta carrera es una de las en que el propio ingeniero puede avanzar más allá del diseño y darse la satisfacción de verlo funcionando.

Hemos conseguido, no solo diseñar el sistema, sino también llegar a hacer funcionar un \textbf{prototipo}, que demuestra que el diseño es válido. Por parte de Mikel, la \textbf{mecánica} ha sido diseñada, pero no ha sido posible llevarla a cabo debido a los altos \textbf{costes} que conlleva.

Ha sido, además, un proyecto que ha abarcado \textbf{un poco de cada rama} de la informática: electrónica, bases de datos, sistemas operativos, ingeniería del \textit{software}, programación concurrente, lenguajes informáticos, algorítmica, seguridad informática, redes y sistemas empotrados. Esto me ha permitido \textbf{desplegar al máximo} mis habilidades y desarrollar \textbf{nuevas capacidades}, como diseño en 3D o programación para la \textit{web}.

La \textbf{interfaz de usuario} obtenida es simple pero cubre completamente las necesidades básicas. Queda pendiente un estudio de mercado y una entrevista con el usuario final para poder ofrecer una \textbf{aplicación completa}. Con la esperanza de que esto suceda, el diseño ha sido \textbf{modular} y la implementación es limpia y está lo suficientemente bien \textbf{documentada} como para poder ampliarla sin problemas.

No ha sido un trabajo sencillo, pero cada minuto que hemos dedicado ha merecido la pena, máxime al fusionar los dos proyectos hermanos y lograr dar vida al \textit{hardware} de la \acrshort{PCB}. Me siento honestamente satisfecho del trabajo realizado, y tanto mi compañero Mikel como yo estamos animados a continuar esta línea de desarrollo.

Reitero mi agradecimiento a mi compañero, a mis tutores, y a todos los que nos han ayudado a llevar este proyecto adelante.

\newpage

\section{Vistas al futuro}

Refiriéndonos al objetivo completo, como hicimos en un principio, hemos alcanzado un alto nivel de desarrollo en cuanto al \textit{software} y la \textit{electrónica} del sistema, dejando pendiente la implantación de la \textbf{mecánica}.

Como músico y como ingeniero puedo asegurar que este invento \textbf{no pretende sustituir} a ningún organista. Estamos convencidos de que el proyecto puede resultar interesante en iglesias con horarios amplios de visitas, o aquellas en las que no suela haber un organista.

Como todo proyecto vivo, siempre surgen tareas e \textbf{ideas nuevas}, algunas de ellas se pueden aplicar inmediatamente, otras deben esperar un tiempo, tal vez a una \textbf{nueva iteración} del desarrollo. Los \textbf{pasos siguientes} para una nueva versión podrían ser:

\begin{enumerate}
	\item Crear una interfaz específica para \textbf{dispositivos móviles}.
	\item Dar al usuario \textbf{mayor control sobre el mando}, contemplar que tengamos varios transmisores, con distinto número de botones, y permitir asignar tanto listas como acciones sobre el reproductor.
	\item Añadir el control de \textbf{energía del órgano} al sistema, permitiendo encenderlo y apagarlo remotamente.
	\item Especificar la \textbf{correspondencia} entre notas musicales y \textbf{registros}, para obtener nuestra propia asignación.
	\item Replantear la compatibilidad con \acrshort{MIDI} y estudiar la posibilidad de producir \textbf{partituras exclusivas}, abriendo una posible vía de \textbf{ingresos económicos}.
\end{enumerate}

Como paso previo, es necesario \textbf{presupuestar} el desarrollo, hacer un \textbf{estudio de mercado} para conocer la viabilidad comercial del proyecto y \textbf{rentabilizar} todo el proceso, pues las próximas tareas que tenemos que llevar a cabo suponen un \textbf{alto coste} económico.

\newpage
\clearpage{\pagestyle{empty}\cleardoublepage}