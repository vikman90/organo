
\chapter{Conclusión futuras}

Finalmente, la consecución de todas las fases descritas inicialmente y la unión de la parte hardware y software del proyecto, ha dado lugar a un sistema de caracterización de dispositivos magnetorresistivos completo, preciso y eficiente.

%Hemos conseguido sacar partido a un conjunto de instrumentos y equipos electrónicos a fin de establecer, en el laboratorio L4, un entorno de trabajo propicio para investigación y caracterización de este tipo de estructuras magnetorresistentes. Se ha colocado por tanto la primera piedra para continuar progresando en el análisis y la investigación de las estructuras AMR, GMR y CMR. Ahora disfrutamos de la posibilidad de llevar a cabo nuestros propios tests de pruebas, pudiendo con ello dar lugar a futuras publicaciones y artículos de interés sobre esta materia. 
%
%El software implementado permite el control unificado de cualquier equipo con el que se quiera medir en el laboratorio, facilitando a cualquier usuario su curva de aprendizaje en lo que a manejo y utilización de la instrumentación involucrada se refiere. No es necesario por tanto que aprendamos a manejar los paneles frontales de los equipo ni su programación, sino simplemente leer el manual de la herramienta software final.
%
%La flexibilidad del software es otro punto a destacar, pudiendo compartir los resultados con otras instituciones que de forma paralela llevan sus propias investigaciones pero que colaboran estrechamente con la Universidad de Granada.
%
%Este proyecto de investigación abre además varios frentes de mejora. La caracterización de los dispositivos magnetorresistivos contempla un amplio abanico de factores influyentes en su comportamiento, entre los que se encuentran la temperatura y el ruido. Ambos factores se presentan a día de hoy como las principales líneas futuras de este sistema.
%
%
%\begin{itemize}
	%\item [] \textbf{Control de la temperatura.}
%\end{itemize}
  %
%Se pretende añadir un módulo de variación de temperatura. Consistirá básicamente en una plataforma de calentamiento y un criostato con el que descender la temperatura de trabajo del dispositivo.
%
%La plataforma de calentamiento consistirá en una plancha de aluminio similar a un \textit{wafer stage}, con un engraving en espiral donde se coloca un \textit{thermocoax} o hilo resistivo de alta disipación térmica que incrementa la temperatura de forma homogénea sobre la plancha, Figura \ref{fig:plat_calent_fig}. Se emplea una pt100 de platino acoplada a la plancha, cuya resistencia indicará el valor de temperatura aproximado.  	
%
%\begin{figure}[H]%here
%\noindent \begin{centering}
%\subfloat[Diseño en Autocad.]{\includegraphics[scale=0.3]{capitulo10/diseno_espiral_acadd}}
%\hspace{0.1cm}
%\subfloat[Plancha de aluminio fresada.]{\includegraphics[scale=0.06]{capitulo10/espiral_grabada_acad}}
%\vspace{0.5cm}
%\smallskip
%\caption{\label{fig:plat_calent_fig} Plataforma de calentamiento con pt100.}
%\par\end{centering}
%\end{figure}
%
%El criostato por su parte dispondrá de una conexión a la toma de agua y se empleará para estudiar el comportamiento del dispositivo a temperaturas inferiores a la ambiente.
%
%El software y su interfaz se completarán para proveer al usuario de determinadas opciones de barridos en temperatura, realizando lecturas en tiempo real de la temperatura del dispositivo por medio de un termómetro IR.
%
%
%\begin{itemize}
	%\item [] \textbf{Estudio de la afectación por ruido.}
%\end{itemize}
%
%De forma adicional, se llevará a cabo un estudio de afectación del ruido en el comportamiento de magnetorresistencias. Se contemplará la fabricación y utilización de una cámara de Faraday que encierre al dispositivo durante su caracterización, para aislarlo de forma completa y estudiar la afectación de radiaciones presentes en el entorno próximo. 
%
%\newpage
%
%Por último, quisiera expresar mi entera satisfacción por el trabajo realizado y el resultado conseguido. El aprendizaje llevado a cabo ha sido muy transversal, permitiéndome trabajar en diferentes disciplinas dentro de un enfoque tanto hardware como software. El diseño estructural, la simulación funcional y fabricación de elementos como las bobinas de Helmholtz o las mesas de puntas auxiliares han supuesto para mí un aumento de la capacidad de ingenio y de reutilización de materiales. 
%
%\cleardoublepage{}
