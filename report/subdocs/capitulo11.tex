
\chapter{Conclusiones}

Es de remarcar que la realizaci�n de las diversas fases que han sido
tratadas en este proyecto ha dado lugar a un producto real que se
encuentra en el mercado actualmente en versi�n demostrativa y que
tiene prevista una implementaci�n real en un corto plazo. El sistema
desarrollado se puede observar en la figura \ref{fig:proyecto}.\\


As�, el sistema descrito en esta memoria y dise�ado como Proyecto
Fin de Carrera, supone un trabajo de desarrollo de ingenier�a dentro
del esquema de trabajo de una empresa real, en el cual se ha otorgado
a dispositivos que presentan movilidad ocasional la posibilidad de
localizarlos geogr�ficamente, obtener informaci�n para su control
y la posibilidad de gestionarlos, todo esto minimizando el coste econ�mico
del servicio. \\


Si bien ya exist�an empresas que se dedicaban a ofrecer este tipo
de servicio, �stas no estaban destinadas espec�ficamente al control
de este tipo de dispositivos, sino m�s bien a la gesti�n de flotas
de dispositivos m�viles en tiempo real. Dentro de este �mbito, las
caracter�sticas que hacen este sistema un producto novedoso en el
sector son: 
\begin{itemize}
\item Es un producto que no est� orientado a la monitorizaci�n en tiempo
real, caracter�stica principal que presentan el resto de sistemas
existentes. Est� \textbf{orientado a} aquellos \textbf{dispositivos}
que se encuentran \textbf{inm�viles} en una posici�n \textbf{durante
un periodo de tiempo determinado}.
\item \textbf{El proyecto minimiza el coste econ�mico y est� optimizado}
para este tipo de aplicaciones, cosa que pese a que se pod�a extender
de los sistemas m�s complejos existentes, no resultaba econ�mico ni
pr�ctico para el usuario final.
\item Ha sido \textbf{dise�ado en base a una arquitectura modular f�cilmente
escalable} para futuras aplicaciones que partiendo de la misma base,
necesiten unas mayores prestaciones. 
\item La aplicaci�n \textbf{permite} no solo la monitorizaci�n de posiciones
geogr�ficas y lectura de datos obtenidos por sensores, est� pensada
para poder dar soporte a \textbf{tareas de gesti�n de la flota} de
dispositivos.
\end{itemize}
~

%
\begin{figure}
\begin{centering}
\includegraphics[scale=0.5]{imagenes/resumen}
\par\end{centering}

\caption{\label{fig:proyecto}Algunas de las partes del proyecto desarrollado.}



\end{figure}


No menos importante quisiera mostrar mi satisfacci�n a nivel personal
por el hecho de haber realizado un proyecto en el que se ha podido
trabajar minuciosamente en cuanto a desarrollo de un producto novedoso
en el que se ha necesitado un desarrollo a nivel software, firmware
y de comunicaciones. \\


Este proyecto supone una formaci�n y un aprendizaje t�cnico totalmente
v�lido para afrontar retos del �mbito laboral y profesional. Adem�s
ilustran el proceso que se ha de seguir para conseguir ubicar un producto
competente dentro del mercado. 


