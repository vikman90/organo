\addcontentsline{toc}{chapter}{\numberline{}{Anexos}}
\appendix
%\setcounter{chapter}{0}\renewcommand{\chaptername}{Appendix} \renewcommand{\theequation}{\Alph{chapter}.\arabic{section}.\arabic{equation}} \addcontentsline{toc}{chapter}{\numberline{}{Anexos}}
%\setcounter{equation}{0}

\chapter[Orden INT/316/2011 sobre seguridad privada]{Orden INT/316/2011, de 1 de febrero, sobre funcionamiento de los sistemas de alarma en el ámbito de la seguridad privada.}
\label{cap:reglamento_seguridad}
\includepdf[pages=-,link=true,linkname=RunSheet]{documentacion/anexos/ley_seguridad_privada.pdf}

\chapter[Planos de zonas no seguras]{Planos de las posibles zonas accesibles por un intruso sin que se avise a la policía}
%\label{cap: planos_zonas_no_seguras}
%\section{Sin seguridad perimetral}
%\label{cap:planos_sin_seg_perimetral}
%
%A continuación se presenta una serie de planos en los que se ven reflejadas las zonas por las que podría transitar un intruso que accediera a la vivienda por todas las puertas y ventanas posibles sin que la central receptora de alarmas pudiera llamar a las Fuerzas y Cuerpos de Seguridad, de acuerdo con la normativa vigente (podrán avisar a dichos Cuerpos de Seguridad cuando se activen tres o más sensores diferentes en un espacio de tiempo limitado).
%\newline
%Se han sombreado con colores diferentes las zonas no seguras accediendo por distintos lugares de la casa, tal y como se indica en la figura \ref{fig:diagrama_colores}.
%\begin{figure}[H]%here
%\noindent \begin{centering}
%\includegraphics[scale=0.8]{diagramas/Diagrama_colores.pdf}
%\par\end{centering}
%\caption{\label{fig:diagrama_colores}Zonas no seguras según la vía de acceso.}
%\end{figure}
%
%\subsection{Zonas no seguras: primer caso}
%
%\includepdf[pages=-,link=true,linkname=RunSheet]{Recorridos_intruso_sin_seguridad_perimetral/Recorrido1/1.pdf}
%\includepdf[pages=-,link=true,linkname=RunSheet]{Recorridos_intruso_sin_seguridad_perimetral/Recorrido1/2.pdf}
%\includepdf[pages=-,link=true,linkname=RunSheet]{Recorridos_intruso_sin_seguridad_perimetral/Recorrido1/3.pdf}
%\includepdf[pages=-,link=true,linkname=RunSheet]{Recorridos_intruso_sin_seguridad_perimetral/Recorrido1/4.pdf}
%
%\subsection{Zonas no seguras: segundo caso}
%
%\includepdf[pages=-,link=true,linkname=RunSheet]{Recorridos_intruso_sin_seguridad_perimetral/Recorrido2/5.pdf}
%\includepdf[pages=-,link=true,linkname=RunSheet]{Recorridos_intruso_sin_seguridad_perimetral/Recorrido2/6.pdf}
%\includepdf[pages=-,link=true,linkname=RunSheet]{Recorridos_intruso_sin_seguridad_perimetral/Recorrido2/7.pdf}
%\includepdf[pages=-,link=true,linkname=RunSheet]{Recorridos_intruso_sin_seguridad_perimetral/Recorrido2/8.pdf}
%
%\subsection{Zonas no seguras: tercer caso}
%
%\includepdf[pages=-,link=true,linkname=RunSheet]{Recorridos_intruso_sin_seguridad_perimetral/Recorrido3/9.pdf}
%\includepdf[pages=-,link=true,linkname=RunSheet]{Recorridos_intruso_sin_seguridad_perimetral/Recorrido3/10.pdf}
%\includepdf[pages=-,link=true,linkname=RunSheet]{Recorridos_intruso_sin_seguridad_perimetral/Recorrido3/11.pdf}
%\includepdf[pages=-,link=true,linkname=RunSheet]{Recorridos_intruso_sin_seguridad_perimetral/Recorrido3/12.pdf}
%
%\subsection{Zonas no seguras: cuarto caso}
%
%\includepdf[pages=-,link=true,linkname=RunSheet]{Recorridos_intruso_sin_seguridad_perimetral/Recorrido4/13.pdf}
%\includepdf[pages=-,link=true,linkname=RunSheet]{Recorridos_intruso_sin_seguridad_perimetral/Recorrido4/14.pdf}
%\includepdf[pages=-,link=true,linkname=RunSheet]{Recorridos_intruso_sin_seguridad_perimetral/Recorrido4/15.pdf}
%\includepdf[pages=-,link=true,linkname=RunSheet]{Recorridos_intruso_sin_seguridad_perimetral/Recorrido4/16.pdf}
%
%\subsection{Zonas no seguras: quinto caso}
%
%\includepdf[pages=-,link=true,linkname=RunSheet]{Recorridos_intruso_sin_seguridad_perimetral/Recorrido5/17.pdf}
%\includepdf[pages=-,link=true,linkname=RunSheet]{Recorridos_intruso_sin_seguridad_perimetral/Recorrido5/18.pdf}
%\includepdf[pages=-,link=true,linkname=RunSheet]{Recorridos_intruso_sin_seguridad_perimetral/Recorrido5/19.pdf}
%\includepdf[pages=-,link=true,linkname=RunSheet]{Recorridos_intruso_sin_seguridad_perimetral/Recorrido5/20.pdf}
%
%\subsection{Zonas no seguras: sexto caso}
%
%\includepdf[pages=-,link=true,linkname=RunSheet]{Recorridos_intruso_sin_seguridad_perimetral/Recorrido6/21.pdf}
%\includepdf[pages=-,link=true,linkname=RunSheet]{Recorridos_intruso_sin_seguridad_perimetral/Recorrido6/22.pdf}
%\includepdf[pages=-,link=true,linkname=RunSheet]{Recorridos_intruso_sin_seguridad_perimetral/Recorrido6/23.pdf}
%\includepdf[pages=-,link=true,linkname=RunSheet]{Recorridos_intruso_sin_seguridad_perimetral/Recorrido6/24.pdf}
%
%\subsection{Zonas no seguras: séptimo caso}
%
%\includepdf[pages=-,link=true,linkname=RunSheet]{Recorridos_intruso_sin_seguridad_perimetral/Recorrido7/25.pdf}
%\includepdf[pages=-,link=true,linkname=RunSheet]{Recorridos_intruso_sin_seguridad_perimetral/Recorrido7/26.pdf}
%\includepdf[pages=-,link=true,linkname=RunSheet]{Recorridos_intruso_sin_seguridad_perimetral/Recorrido7/27.pdf}
%\includepdf[pages=-,link=true,linkname=RunSheet]{Recorridos_intruso_sin_seguridad_perimetral/Recorrido7/28.pdf}
%
%\subsection{Zonas no seguras: octavo caso}
%
%\includepdf[pages=-,link=true,linkname=RunSheet]{Recorridos_intruso_sin_seguridad_perimetral/Recorrido8/29.pdf}
%\includepdf[pages=-,link=true,linkname=RunSheet]{Recorridos_intruso_sin_seguridad_perimetral/Recorrido8/30.pdf}
%\includepdf[pages=-,link=true,linkname=RunSheet]{Recorridos_intruso_sin_seguridad_perimetral/Recorrido8/31.pdf}
%\includepdf[pages=-,link=true,linkname=RunSheet]{Recorridos_intruso_sin_seguridad_perimetral/Recorrido8/32.pdf}
%
%\newpage
%\section{Con seguridad perimetral}
%\label{cap:planos_con_seg_perimetral}
%
%%-------------------------------------------------------------------------------------%
%
%A continuación veremos una mejora introducida al caso anterior: la seguridad perimetral. Con este tipo de seguridad se minimizan las zonas accesibles por el intruso, tal y como se observa en los planos, en los que se ven reflejadas las zonas por las que podría transitar un intruso que accediera a la vivienda por todas las puertas y ventanas posibles sin que la central receptora de alarmas pudiera llamar a las Fuerzas y Cuerpos de Seguridad, de acuerdo con la normativa vigente (podrán avisar a dichos Cuerpos de Seguridad cuando se activen tres o más sensores diferentes en un espacio de tiempo limitado).
%\newline
%Se han sombreado con colores diferentes las zonas no seguras accediendo por distintos lugares de la casa, tal y como se indica en la figura \ref{fig:diagrama_colores}.
%\begin{figure}[H]%here
%\noindent \begin{centering}
%\includegraphics[scale=0.8]{diagramas/Diagrama_colores_seg_perimetral.pdf}
%\par\end{centering}
%\caption{\label{fig:diagrama_colores_perimetral}Zonas no seguras según la vía de acceso (con seguridad perimetral).}
%\end{figure}
%
%
%\subsection{Zonas no seguras: primer caso}
%
%\includepdf[pages=-,link=true,linkname=RunSheet]{Recorridos_intruso_con_seguridad_perimetral/Recorrido1/33.pdf}
%\includepdf[pages=-,link=true,linkname=RunSheet]{Recorridos_intruso_con_seguridad_perimetral/Recorrido1/34.pdf}
%\includepdf[pages=-,link=true,linkname=RunSheet]{Recorridos_intruso_con_seguridad_perimetral/Recorrido1/35.pdf}
%\includepdf[pages=-,link=true,linkname=RunSheet]{Recorridos_intruso_con_seguridad_perimetral/Recorrido1/36.pdf}
%
%\subsection{Zonas no seguras: segundo caso}
%
%\includepdf[pages=-,link=true,linkname=RunSheet]{Recorridos_intruso_con_seguridad_perimetral/Recorrido2/37.pdf}
%\includepdf[pages=-,link=true,linkname=RunSheet]{Recorridos_intruso_con_seguridad_perimetral/Recorrido2/38.pdf}
%\includepdf[pages=-,link=true,linkname=RunSheet]{Recorridos_intruso_con_seguridad_perimetral/Recorrido2/39.pdf}
%\includepdf[pages=-,link=true,linkname=RunSheet]{Recorridos_intruso_con_seguridad_perimetral/Recorrido2/40.pdf}
%
%\subsection{Zonas no seguras: tercer caso}
%
%\includepdf[pages=-,link=true,linkname=RunSheet]{Recorridos_intruso_con_seguridad_perimetral/Recorrido3/41.pdf}
%\includepdf[pages=-,link=true,linkname=RunSheet]{Recorridos_intruso_con_seguridad_perimetral/Recorrido3/42.pdf}
%\includepdf[pages=-,link=true,linkname=RunSheet]{Recorridos_intruso_con_seguridad_perimetral/Recorrido3/43.pdf}
%\includepdf[pages=-,link=true,linkname=RunSheet]{Recorridos_intruso_con_seguridad_perimetral/Recorrido3/44.pdf}
%
%\subsection{Zonas no seguras: cuarto caso}
%
%\includepdf[pages=-,link=true,linkname=RunSheet]{Recorridos_intruso_con_seguridad_perimetral/Recorrido4/45.pdf}
%\includepdf[pages=-,link=true,linkname=RunSheet]{Recorridos_intruso_con_seguridad_perimetral/Recorrido4/46.pdf}
%\includepdf[pages=-,link=true,linkname=RunSheet]{Recorridos_intruso_con_seguridad_perimetral/Recorrido4/47.pdf}
%\includepdf[pages=-,link=true,linkname=RunSheet]{Recorridos_intruso_con_seguridad_perimetral/Recorrido4/48.pdf}
%
%\subsection{Zonas no seguras: quinto caso}
%
%\includepdf[pages=-,link=true,linkname=RunSheet]{Recorridos_intruso_con_seguridad_perimetral/Recorrido5/49.pdf}
%\includepdf[pages=-,link=true,linkname=RunSheet]{Recorridos_intruso_con_seguridad_perimetral/Recorrido5/50.pdf}
%\includepdf[pages=-,link=true,linkname=RunSheet]{Recorridos_intruso_con_seguridad_perimetral/Recorrido5/51.pdf}
%\includepdf[pages=-,link=true,linkname=RunSheet]{Recorridos_intruso_con_seguridad_perimetral/Recorrido5/52.pdf}
%
%\subsection{Zonas no seguras: sexto caso}
%
%\includepdf[pages=-,link=true,linkname=RunSheet]{Recorridos_intruso_con_seguridad_perimetral/Recorrido6/53.pdf}
%\includepdf[pages=-,link=true,linkname=RunSheet]{Recorridos_intruso_con_seguridad_perimetral/Recorrido6/54.pdf}
%\includepdf[pages=-,link=true,linkname=RunSheet]{Recorridos_intruso_con_seguridad_perimetral/Recorrido6/55.pdf}
%\includepdf[pages=-,link=true,linkname=RunSheet]{Recorridos_intruso_con_seguridad_perimetral/Recorrido6/56.pdf}
%
%\subsection{Zonas no seguras: séptimo caso}
%
%\includepdf[pages=-,link=true,linkname=RunSheet]{Recorridos_intruso_con_seguridad_perimetral/Recorrido7/57.pdf}
%\includepdf[pages=-,link=true,linkname=RunSheet]{Recorridos_intruso_con_seguridad_perimetral/Recorrido7/58.pdf}
%\includepdf[pages=-,link=true,linkname=RunSheet]{Recorridos_intruso_con_seguridad_perimetral/Recorrido7/59.pdf}
%\includepdf[pages=-,link=true,linkname=RunSheet]{Recorridos_intruso_con_seguridad_perimetral/Recorrido7/60.pdf}
%
%\subsection{Zonas no seguras: octavo caso}
%
%\includepdf[pages=-,link=true,linkname=RunSheet]{Recorridos_intruso_con_seguridad_perimetral/Recorrido8/61.pdf}
%\includepdf[pages=-,link=true,linkname=RunSheet]{Recorridos_intruso_con_seguridad_perimetral/Recorrido8/62.pdf}
%\includepdf[pages=-,link=true,linkname=RunSheet]{Recorridos_intruso_con_seguridad_perimetral/Recorrido8/63.pdf}
%\includepdf[pages=-,link=true,linkname=RunSheet]{Recorridos_intruso_con_seguridad_perimetral/Recorrido8/64.pdf}
%%-------------------------------------------------------------------------------------%
%
%
%\newpage
%\section{Con seguridad perimetral: sensores activos según recorrido}
%\label{cap:planos_con_seg_perimetral_sensores}
%
%%-------------------------------------------------------------------------------------%
%
%A continuación se presentan los distintos recorridos que puede realizar un intruso con los diferentes sensores por los que va siendo detectado.
%\newline
%Se ha sombreado con colores diferentes según haya sido detectado el intruso por uno, dos o tres sensores, tal y como se indica en la figura \ref{fig:diagrama_colores_sensores}.
%\newline
%Nota: como mínimo habrá sido detectado por un detector, ya que al cruzar la barrera perimetral se activará al menos uno de sus sensores.
%\begin{figure}[H]%here
%\noindent \begin{centering}
%\includegraphics[scale=0.8]{diagramas/Diagrama_colores_sensores.pdf}
%\par\end{centering}
%\caption{\label{fig:diagrama_colores_sensores}Sensores activos según recorrido (con seguridad perimetral).}
%\end{figure}
%
%\subsection{Zonas no seguras: primer caso}
%
%\includepdf[pages=-,link=true,linkname=RunSheet]{Recorridos_sensores_activos/Recorrido1/65.pdf}
%\includepdf[pages=-,link=true,linkname=RunSheet]{Recorridos_sensores_activos/Recorrido1/66.pdf}
%\includepdf[pages=-,link=true,linkname=RunSheet]{Recorridos_sensores_activos/Recorrido1/67.pdf}
%\includepdf[pages=-,link=true,linkname=RunSheet]{Recorridos_sensores_activos/Recorrido1/68.pdf}
%
%\subsection{Zonas no seguras: segundo caso}
%
%\includepdf[pages=-,link=true,linkname=RunSheet]{Recorridos_sensores_activos/Recorrido2/69.pdf}
%\includepdf[pages=-,link=true,linkname=RunSheet]{Recorridos_sensores_activos/Recorrido2/70.pdf}
%\includepdf[pages=-,link=true,linkname=RunSheet]{Recorridos_sensores_activos/Recorrido2/71.pdf}
%\includepdf[pages=-,link=true,linkname=RunSheet]{Recorridos_sensores_activos/Recorrido2/72.pdf}
%
%\subsection{Zonas no seguras: tercer caso}
%
%\includepdf[pages=-,link=true,linkname=RunSheet]{Recorridos_sensores_activos/Recorrido3/73.pdf}
%\includepdf[pages=-,link=true,linkname=RunSheet]{Recorridos_sensores_activos/Recorrido3/74.pdf}
%\includepdf[pages=-,link=true,linkname=RunSheet]{Recorridos_sensores_activos/Recorrido3/75.pdf}
%\includepdf[pages=-,link=true,linkname=RunSheet]{Recorridos_sensores_activos/Recorrido3/76.pdf}
%
%\subsection{Zonas no seguras: cuarto caso}
%
%\includepdf[pages=-,link=true,linkname=RunSheet]{Recorridos_sensores_activos/Recorrido4/77.pdf}
%\includepdf[pages=-,link=true,linkname=RunSheet]{Recorridos_sensores_activos/Recorrido4/78.pdf}
%\includepdf[pages=-,link=true,linkname=RunSheet]{Recorridos_sensores_activos/Recorrido4/79.pdf}
%\includepdf[pages=-,link=true,linkname=RunSheet]{Recorridos_sensores_activos/Recorrido4/80.pdf}
%
%\subsection{Zonas no seguras: quinto caso}
%
%\includepdf[pages=-,link=true,linkname=RunSheet]{Recorridos_sensores_activos/Recorrido5/81.pdf}
%\includepdf[pages=-,link=true,linkname=RunSheet]{Recorridos_sensores_activos/Recorrido5/82.pdf}
%\includepdf[pages=-,link=true,linkname=RunSheet]{Recorridos_sensores_activos/Recorrido5/83.pdf}
%\includepdf[pages=-,link=true,linkname=RunSheet]{Recorridos_sensores_activos/Recorrido5/84.pdf}
%
%\subsection{Zonas no seguras: sexto caso}
%
%\includepdf[pages=-,link=true,linkname=RunSheet]{Recorridos_sensores_activos/Recorrido6/85.pdf}
%\includepdf[pages=-,link=true,linkname=RunSheet]{Recorridos_sensores_activos/Recorrido6/86.pdf}
%\includepdf[pages=-,link=true,linkname=RunSheet]{Recorridos_sensores_activos/Recorrido6/87.pdf}
%\includepdf[pages=-,link=true,linkname=RunSheet]{Recorridos_sensores_activos/Recorrido6/88.pdf}
%
%\subsection{Zonas no seguras: séptimo caso}
%
%\includepdf[pages=-,link=true,linkname=RunSheet]{Recorridos_sensores_activos/Recorrido7/89.pdf}
%\includepdf[pages=-,link=true,linkname=RunSheet]{Recorridos_sensores_activos/Recorrido7/90.pdf}
%\includepdf[pages=-,link=true,linkname=RunSheet]{Recorridos_sensores_activos/Recorrido7/91.pdf}
%\includepdf[pages=-,link=true,linkname=RunSheet]{Recorridos_sensores_activos/Recorrido7/92.pdf}
%
%\subsection{Zonas no seguras: octavo caso}
%
%\includepdf[pages=-,link=true,linkname=RunSheet]{Recorridos_sensores_activos/Recorrido8/93.pdf}
%\includepdf[pages=-,link=true,linkname=RunSheet]{Recorridos_sensores_activos/Recorrido8/94.pdf}
%\includepdf[pages=-,link=true,linkname=RunSheet]{Recorridos_sensores_activos/Recorrido8/95.pdf}
%\includepdf[pages=-,link=true,linkname=RunSheet]{Recorridos_sensores_activos/Recorrido8/96.pdf}
%
%\subsection{Zonas no seguras: noveno caso}
%
%\includepdf[pages=-,link=true,linkname=RunSheet]{Recorridos_sensores_activos/Recorrido9/97.pdf}
%\includepdf[pages=-,link=true,linkname=RunSheet]{Recorridos_sensores_activos/Recorrido9/98.pdf}
%\includepdf[pages=-,link=true,linkname=RunSheet]{Recorridos_sensores_activos/Recorrido9/99.pdf}
%\includepdf[pages=-,link=true,linkname=RunSheet]{Recorridos_sensores_activos/Recorrido9/100.pdf}
%
%\subsection{Zonas no seguras: décimo caso}
%
%\includepdf[pages=-,link=true,linkname=RunSheet]{Recorridos_sensores_activos/Recorrido10/101.pdf}
%\includepdf[pages=-,link=true,linkname=RunSheet]{Recorridos_sensores_activos/Recorrido10/102.pdf}
%\includepdf[pages=-,link=true,linkname=RunSheet]{Recorridos_sensores_activos/Recorrido10/103.pdf}
%\includepdf[pages=-,link=true,linkname=RunSheet]{Recorridos_sensores_activos/Recorrido10/104.pdf}
%
%\subsection{Zonas no seguras: undécimo caso}
%
%\includepdf[pages=-,link=true,linkname=RunSheet]{Recorridos_sensores_activos/Recorrido11/105.pdf}
%\includepdf[pages=-,link=true,linkname=RunSheet]{Recorridos_sensores_activos/Recorrido11/106.pdf}
%\includepdf[pages=-,link=true,linkname=RunSheet]{Recorridos_sensores_activos/Recorrido11/107.pdf}
%\includepdf[pages=-,link=true,linkname=RunSheet]{Recorridos_sensores_activos/Recorrido11/108.pdf}
%
%\subsection{Zonas no seguras: duodécimo caso}
%
%\includepdf[pages=-,link=true,linkname=RunSheet]{Recorridos_sensores_activos/Recorrido12/109.pdf}
%\includepdf[pages=-,link=true,linkname=RunSheet]{Recorridos_sensores_activos/Recorrido12/110.pdf}
%\includepdf[pages=-,link=true,linkname=RunSheet]{Recorridos_sensores_activos/Recorrido12/111.pdf}
%\includepdf[pages=-,link=true,linkname=RunSheet]{Recorridos_sensores_activos/Recorrido12/112.pdf}
%%-------------------------------------------------------------------------------------%
%
%
%
%
%
%
%
%%bibliografía es del datasheet de HIRK 433
%
%%\begin{figure}[H]
%%\noindent \begin{centering}
%%\includegraphics[scale=0.3]{imagenes/HIRK_433_photo}
%%\par\end{centering}
%%\caption{\label{fig:Receptor y Decodificador HIRK}Tarjeta receptora y decodificadora HIRK.}
%%\end{figure}
%%
%%\textbf{Características:}
%%
%%\begin{itemize}
	%%\item Decodificador avanzado Keeloq.
	%%\item Rango de más 100 metros.
	%%\item Fácil aprendizaje del transmisor.
	%%\item Salidas, momentáneas o fijas, y datos serie.
	%%\item Led de conducción directa que indica la recepción de datos.
	%%\item Alimentación a 5 V.
	%%\item Memoria Flash re-programable.\\
%%\end{itemize}
%%
%%
%%\textbf{Circuito para la aplicación:}\\
%%
%%El circuito que hemos seguido para la adaptación del HIRK a nuestra aplicación es el mostrado en la figura A2. Donde  \textit{``LK1''} y \textit{``LK2''} posibilitan los distintos funcionamientos del dispositivo, dependiendo de su configuración. Según la tabla mostrada en la figura A3.
%%
 %%\begin{figure}[H]
%%\noindent \begin{centering}
%%\includegraphics[scale=0.47]{imagenes/circuitoHIRK}
%%\par\end{centering}
%%\caption{\label{fig:Ciruito para la aplicación de HIRK}Circuito para el correcto funcionamiento del HIRK.}
%%\end{figure} 
%%
%%\textbf{Posibles salidas digitales:}
%%
%%\begin{figure}[H]
%%\noindent \begin{centering}
%%\includegraphics[scale=0.6]{imagenes/SalidasVSlink}
%%\par\end{centering}
%%\caption{\label{fig:Posibles Salidas dependiendo de la posición de los Link}Posibles Salidas dependiendo de la posición de los Link\\\textit{Mom: salida activa durante la duración del pulso.}\\\textit{Latch: salida fija, que cambia de estado con cada nuevo pulso.}\\}
%%\end{figure}
%%
%%
%%
%%
%%
%%\textbf{Para aprender un nuevo Mando:}
%%
%%\begin{enumerate}
	%%\item Pulsar brevemente el botón de aprendizaje (Learn SW).
	%%\item El led de aprendizaje se iluminará.
	%%\item Presionar cualquier botón del mando una vez. El Led de aprendizaje se apagará.
	%%\item Presionar el botón de nuevo. El Led de aprendizaje parpadeará (aprox. 10 s).
	%%\item Esperar a que el Led de aprendizaje deje de parpadear.
	%%\item Mando memorizado.\\
%%\end{enumerate}
%%
%%\textbf{Salida de Datos Serie:}\\
%%
%%El decodificador HiRK puede configurarse para obtener los datos en serie por la salida OP4 (opción seleccionable). Esta salida
%%proporciona el número de serie, tecla y estado de la batería del codificador transmisor que ha recibido con éxito. Estos datos tienen un estado de inactividad con un "1" lógico. Los datos en serie se emiten continuamente mientras se está recibiendo desde el transmisor. Es decir, este resultado es válido
%%independientemente de si el decodificador se ha aprendido al codificador o no. Si el usuario requiere del número serie del codificador, éste deberá ser previamente aprendido.
%%Los datos recibidos por el decodificador (desde codificador) son chequeados por un código Manchester para así asegurar su validez (no es un descifrado KeeLoq) y si éstos son correctos, se emiten por el pin de salida de datos serie (OP4):
%%
%%\begin{itemize}
	%%\item El número de serie KeeLoq (7 dígitos).
	%%\item Los datos del botón y el estado de labatería del transmisor.
	%%\item Un carácter de retorno de carro y otro de fin de línea.\\
%%\end{itemize}
%%
%%\textbf{Formato de los Datos Serie:}\\
%%
%%Los datos serie son enviados como un flujo de caracteres ASCII a 9600 baudios. Y el formato del carácter es de 8 bits de datos con un bit de stop, sin paridad.
%%
%%Una cadena de caracteres es enviada cada vez que se recibe un paquete válido desde el transmisor. Debido a la naturaleza del paquete KeeLoq habrá un mínimo intervalo entre cada cadena de caracteres de 150 milisegundos.
%%
%%Los 10 caracteres son enviados según el formato de la figura \ref{fig:Formato de los caracteres de salida del dispositivo HIRK}:
%%
%%
%%\begin{figure}[H]
%%\noindent \begin{centering}
%%\includegraphics[scale=0.47]{imagenes/formato_datos_serie_foto}
%%\par\end{centering}
%%\caption{\label{fig:Formato de los caracteres de salida del dispositivo HIRK}Salida desde el pin 24 (OP4).}
%%\end{figure}
%%
%%
%%
%%\textbf{El número serie del transmisor:}\\
%%
%%Está compuesto de 7 caracteres ASCII pertenecientes al grupo: 0 1 2 3 4 5 6 7 8 9 A B C D E F. El dígito más significante del número de serie es transmitido primero. Éste grupo proporciona un total de 28 millones de posibles números serie.\\
%%
%%En la tabla \ref{fig:tablaASCII} podemos ver la codificación de los datos ASCII:
%%
%%\begin{table}[H]
		%%\centering			
				%%\includegraphics[width=0.80\textwidth]{imagenes/ascii1}				
			%%\caption{Tabla de caracteres ASCII.}
			%%\label{fig:tablaASCII}
%%\end{table}
%%
%%
%%\textbf{El Botón Keeloq pulsado:}\\
%%
%%El botón pulsado es transmitido como un carácter ASCII en el rango desde \textit{A} hasta \textit{O}, o si se detecta el bit de batería baja del codificador (batería baja en el mando transmisor) entonces el rango es desde \textit{a} hasta \textit{o}. Los bits de datos de Keeloq \textit{S0, S1, S2 y S3} conforman los bits ASCII menos significativos \textit{D0, D1, D2, D3}.
%%
%%Si un transmisor de un solo botón es pulsado, diciendo \textit{S0}, entonces se envía el carácter \textit{A} (o \textit{a} con batería baja).
%%Si el botón \textit{S1} del codificador es pulsado entonces enviará \textit{B} (o \textit{b}).
%%Si \textit{S0} y \textit{S1} son ambos presionados simultáneamente entonces el carácter enviado es el \textit{C} (o \textit{c}).
%%Si todos los botones Keeloq son pulsados simultáneamente entonces se envía el carácter \textit{O} (o el \textit{o}).
%%Como ejemplo: si tenemos un mando de tres botones con número de serie \textit{2345678} el dato serie que tendremos a la salida cuando la transmisión codificada sea decodificada, será el siguiente:
%%
%%\begin{itemize}
	%%\item Si se pulsa \textit{S1}: \textit{2345678A} (o con batería baja \textit{2345678a}).
	%%\item Si se pulsa \textit{S2}: \textit{2345678B} (o con batería baja \textit{2345678b}).
	%%\item Si se presionan simultáneamente \textit{S1} y \textit{S2}: \textit{2345678C} (o \textit{2345678c}).
	%%\item Si se pulsa \textit{S4}: 2345678H (o \textit{2345678h} para batería baja).\\
%%\end{itemize}
%%
%%
%%
%%
%%
%
%
%
%
%
%%\end{document}
%
%%fin prueba migu
%
%
%%\begin{table}[!hbt]
%%\label{tablaejemplo}
%%\begin{center}
%%\begin{tabular}{|r|l|p{5cm}|}
%%\hline
%%\multicolumn{3}{|c|}{Tabla de ejemplo} \\
%%\hline
%%\hline
%%Elemento col 1 & Elemento col 2 & parrafogordo con un montón de texto escrito para rellenar este ejemplo y que se vea cómo funciona esto de las tablas en latex, que es un programa muy chulo \\
%%\hline
%%\end{tabular}
%%\caption[Tabla de Ejemplo]{Esto es el texto al pié de la tabla de ejmemplo de nuestro curso de \LaTeX2e}
%%\end{center}
%%\end{table}
%
%
%
%%imagen a la izquierda
%%Como se estableció durante el diseño, el sistema operativo que se
%%instalará en el PC de sobremesa será Linux, más concretamente la distribución
%%\textbf{Ubuntu 8.10}.\\
%%\begin{wrapfigure}{o}{0.15\columnwidth}%
%%\begin{centering}
%%\includegraphics[scale=0.15]{imagenes/ubuntu8_10}
%%\par\end{centering}
%%
%%\caption{Sistema operativo instalado.}
%
%
%%\end{wrapfigure}%
%
%%imagen a la derecha
%%\begin{wrapfigure}{r}{0.25\columnwidth}%
%%\begin{centering}
%%\includegraphics{imagenes/xampp.jpeg}
%%\par\end{centering}
%
%%\caption{Plataforma XAMPP.}
%%\end{wrapfigure}%
%
%
%
%
%%\begin{figure}[h]
%%\noindent \begin{left}
%%\includegraphics[width=0.45\columnwidth]{imagenes/grafica_signal_recibida}
%%\par\end{left}
%%\caption{\label{fig: Grafico de Alcance} Gráfico de Alcance.}
%%\end{figure}
%
%%
%
%
%%
%%\chapter{Entorno de usuario para fabricación con LPKF BoardMaster}
%%\label{cap:lpkf}
%%
%%
%%\begin{enumerate}
	%%\item \textbf{Empezar a usar el programa BoardMaster}
	%%
	%%Una vez iniciado \textbf{BoardMaster}, una ventana nos pedir\'a que encendamos la m\'aquina LPKF para la creaci\'on de prototipos. El bot\'on de encendido est\'a en la parte lateral de la m\'aquina, as\'i que lo ponemos a 1 y y hacemos clic en \textquotedblleft{}OK\textquotedblright{} en la ventana que nos aparezca. Después, otra ventana aparecer\'a, y desaparecer\'a despu\'es de que la m\'aquina se encienda.
	%%
			%%\begin{figure}[H]
			%%\centering
				%%\includegraphics[width=0.35\textwidth]{imagenes/BM_turnon.png}
				%%\includegraphics[width=0.40\textwidth]{imagenes/BM_turnon2.png}
			%%\label{fig:BM_turnon}
		%%\end{figure}
	%%
	%%\item \textbf{Configurar el sustrato}
	%%
	%%Se fijar\'a el sustrato a la m\'aquina LPKF usando los pines que nos servir\'an como puntos de referencia. Usaremos cinta adhesiva para fijar los bordes del sustrato de manera que \'este no se mueva durante la fabricaci\'on del prototipo.	
	%%
	%%
	%%\item \textbf{Importar el fichero \textbf{.LMD}}
	%%
%%Para la creaci\'on de prototipos de circuitos impresos con el pl\'oter es necesario disponer de un fichero con extensi\'on \textit{.LMD} creado con \textbf{CircuitCAM} previamente. Estos ficheros que se crean durante la exportaci\'on, contienen los datos de producci\'on de todo el circuito impreso.
%%
	%%\item \textbf{Comprobaci\'on de fresas}
	%%
	%%Con esta funci\'on comprobaremos el estado de las brocas (fresas) antes de cada operaci\'on por si es necesario cambiar la herramienta de trabajo. Esta funci\'on resulta muy \'util para que la fabricaci\'on de la placa de PCB se lleve a cabo en condiciones \'optimas.
	%%
			%%\begin{figure}[H]
			%%\centering
				%%\includegraphics[width=0.80\textwidth]{imagenes/fresas.png}
				%%\caption{Brocas para fresado de placa PCB.}
			%%\label{fig:fresas}
		%%\end{figure}
	%%
	%%\item \textbf{Selecci\'on de herramientas}
	%%
	%%Una vez cargado en BoardMaster el fichero \textit{.LMD} podremos visualizar las herramientas (plumas) asignadas a cada una de las fases de fabricaci\'on del circuito impreso pudiendo aqu\'i cambiar la herramienta asignada a cada fase seg\'un los requisitos del prototipo.
			%%
	%%\item \textbf{Establecer dimensiones del material}\label{cap:dimensiones}
	%%
%%Para definir el tama\~no del material, desplazaremos el cabezal de fresa/taladro primero a la esquina anterior derecha y despu\'es a la esquina posterior izquierda. Esto nos permitir\'a establecer el tama\~no del material base que se mostrar\'a en gris oscuro en el \'area de trabajo de BoardMaster.\\
%%
%%Al establecer el tama\~no del material, debemos tener en cuenta la l\'inea de fijaci\'on y el tama\~no del limitador de profundidad de trabajo as\'i como la ubicaci\'on de las plumas.\\
%%
%%\newpage
%%Los pasos a seguir son los siguientes:
%%
%%\begin{itemize}
	%%\item Mover la m\'aquina a los extremos de la placa usando el paso y las flechas. Las 4 flechas indican la direcci\'on del motor y el n\'umero indica la distancia a moverse.
  %%\item Hacer clic en \textit{Configuration/Material/Set Low Corner}.
	%%\item Mover la m\'aquina al extremo diagonal del material y de nuevo usar el cursor y los pasos. Hacer clic en \textit{Configuration/Material/Set High Corner}.
	%%\item Un \'area gris nos muestra la zona del material donde se va a fabricar el prototipo.
%%\end{itemize}
%%
%%NOTA: Asegurarse de que el material est\'a fijado con los pines y con la cinta para prevenir movimientos durante el proceso de fabricaci\'on.
%%
	%%\item \textbf{Biblioteca de herramientas}
	%%
%%En estas bibliotecas est\'an definidas todas las herramientas necesarias para el funcionamiento del pl\'oter de circuitos impresos con sus caracter\'isticas. Aqu\'i se podr\'an modificar par\'ametros tales como el di\'ametro, el tipo de funci\'on del cabezal (\textit{Mill/Drill}), la vida \'util (m\'aximo recorrido de fresado o m\'aximo n\'umero de perforaciones de la herramienta) o la velocidad \'optima (velocidad de giro \'optima del husillo de fresa/taladro).
%%
	%%\item \textbf{Desplazamiento del cabezal}
	%%
	%%Durante la fabricaci\'on del circuito impreso, podremos desplazar el cabezal de fresa/taladro hacia distintas posiciones pertinentes:
	%%
	%%\begin{itemize}
		%%\item \textbf{Home}: lleva el cabezal de fresa/taladro a la posición origen de datos (comprobación del sistema de agujeros pilotos fiducial o creación de uno nuevo).
		%%\item \textbf{Pausa}: lleva el cabezal de fresa/taladro a la posición de pausa (cambio o inversión del material).
		%%\item \textbf{Zero-Position}: lleva el cabezal de fresa/taladro a la posición de cambio de herramienta (para desconectar el plóter de circuitos impresos).
		%%\item \textbf{Cámara$\rightarrow$HEAD}: lleva la cámara a la posición actual del cabezal de fresa/taladro.
		%%\item \textbf{HEAD$\rightarrow$Cámara}: lleva el cabezal de fresa/taladro a la posición actual de la cámara.
	%%\end{itemize}
	%%
	%%\newpage
	%%\item \textbf{Fases de producci\'on}
	%%
	%%Todos los datos que haya que enviar al pl\'oter de circuitos impresos con LPKF BoardMaster deben estar asignados a una determinada fase de producci\'on. \'Estas, que ya est\'an definidas en el archivo de inicializaci\'on, definen una secuencia de trabajo en la que el usuario no necesita intervenir. Sin embargo, se pueden adaptar a las necesidades en cualquier momento.\\
%%
%%Cuando se abre el cuadro de di\'alogo correspondiente, aparecen listadas todas las fases de producci\'on est\'andar definidas. Si se selecciona una fase de producci\'on se pueden visualizar sus par\'ametros, y en su caso, cambiarlos. 
%%
			%%\begin{figure}[H]
			%%\centering
				%%\includegraphics[width=0.90\textwidth]{imagenes/LPKF_fases.png}
				%%\caption{Fases de producci\'on.}
			%%\label{fig:LPKF_fases}
		%%\end{figure}
%%
%%Ser\'an necesarios dos ajustes para cada fase de producci\'on:
%%
%%\begin{itemize}
	%%\item El color en el que presentar los datos asignados a esta fase de producci\'on. Este ajuste sirve para diferenciar las distintas fases de producci\'on en el \'area de trabajo.
	%%\item La definici\'on del lado del circuito impreso (lado de montaje o lado de soldadura) para una fase de producci\'on. Marcaremos el cuadro de verificaci\'on \textbf{cara opuesta} para todas las fases de producci\'on que mecanicen la parte inferior del circuito impreso, mientras que en todas las demas fases de producci\'on en las que no est\'e marcado \'este, se mecanizar\'a el lado de montaje.
%%\end{itemize}
%%\end{enumerate}
%%
%%
%%
%%
%%\chapter{Sistema de Transmisión Keeloq}
%%\label{cap:keeloqGENERAL}
%%\section{Sistema seguro de apertura de puertas a distancia}
%%%bibliografía http://www.ecojoven.com/uno/03/keeloq.html
%%
%%KeeLoq es el sistema seguro de apertura de puertas diseñado por \textit{MicroChip} para la apertura de puertas o cualquier tipo de mando a distancia. Vamos a ver en este artículo cómo funciona este sistema basado en codificación por salto de código. Aunque actualmente este sistema ha evolucionado y es más complejo, este es un buen punto de partida para conocer su funcionamiento.\\
%%
%%KeeLoq es propiamente un proceso de cifrado o encriptación de la información de telecomando que se puede usar con cualquier tipo de mando a distancia, bien sea por infrarrojos o por radio. Cada vez que se transmite una orden de apertura o actuación, se envía una ristra de datos digitales formados por 66 bits. Recuerda que un bit es en realidad un uno o un cero, representados por la presencia o la ausencia de una corriente, un voltaje, un circuito cerrado o abierto... En otras palabras, cada bit representa el estado de un proceso que puede tener sólo dos estados.\\
%%
%%
%%\begin{figure}[H]
%%\noindent \begin{centering}
%%\includegraphics[scale=1]{imagenes/trama_datos_keeloq}
%%\par\end{centering}
%%\caption{\label{fig: Trama de datos Keeloq}Una trama de datos Keeloq.}
%%\end{figure}
%%
%%
%%Como decíamos, cada transmisión con un sistema KeeLoq contiene 66 bits formados por los siguientes códigos:
%%
%%\begin{itemize}
	%%\item 32 bits cifrados formados por:
	%%\begin{itemize}
		%%\item un código de salto generado por un algoritmo no lineal: el algoritmo KeeLoq.
	%%\end{itemize}
	%%\item 34 bits de código fijo formados por:
	%%\begin{itemize}
		%%\item 28 bits que representan el número de serie del codificador.
		%%\item 6 bits de estado, formados a su vez por:
		%%\begin{itemize}
			%%\item 4 bits de función, indicando el estado de los pulsadores.
			%%\item 2 bits de CRC (código de redundancia cíclica) que sirven para verificar que el resto de los datos ha llegado correctamente.
		%%\end{itemize}
		%%
	%%\end{itemize}
	%%
%%\end{itemize}
%%
%%En los sistemas simples de mando a distancia hay dos formas clásicas de conseguir accesos no autorizados: el escaneado de códigos y el robo de códigos.
%%
%%\section{Escaneado de códigos}
%%
%%Muchos sistemas de apertura de puertas utilizan un código único y fijo de 8 bits. Cada vez que se quiere abrir una puerta se envía ese código fijo. Con ocho bits, se consiguen sólo \[2^8= 256  combinaciones\]Si al receptor llega el código de apertura, siempre el mismo, procederá a abrir. Todos los transmisores utilizan un rango de frecuencia estandarizado, el establecido por la normativa del país.
%%
%%Es muy fácil construir un transmisor que probando todas las combinaciones posibles acierte rápidamente con el código correcto. Si el transmisor prueba 8 códigos por segundo, en 32 segundos, como mucho, habrá conseguido abrir la puerta. Con 16 bits de código, podrían tardarse hasta dos horas y media (216 = 65536 códigos). Con 66 bits se alcanzan 7'3 x 1019 códigos. Se tardarían millones de años en probar todas las combinaciones. La primera medida de seguridad del sistema KeeLoq es, por tanto, usar una longitud de código suficientemente larga como para evitar el escaneado de códigos.
%%
%%\section{Robo de código}
%%
%%Con un receptor que guarde el código enviado por un transmisor por radio o por infrarrojos se puede conseguir el código y tener acceso fácilmente. Sólo hace falta esperar que alguien con acceso autorizado abra y robarle el código escuchando su transmisor.
%%
%%El sistema de salto de código KeeLoq jamás transmite dos veces el mismo código, ni siquiera en dos veces su propio tiempo de vida. Cada vez que pulsamos el botón de transmisión se envía un código diferente. Los códigos parecen aleatorios, no hay relación aparente entre dos códigos seguidos. El robo de códigos nunca funcionará.
%%
%%Los circuitos integrados \textit{HCS200, HCS300, HCS301 y HCS360} de \textit{Microchip} como codificadores y el \textit{HCS500} y otros como decodificadores permiten hasta 15 funciones de comandos diferentes. Sólo se necesita añadirles la pila, los botones y el sistema de transmisión por radio o por infrarrojos. Incorporan dentro del chip una memoria EEPROM en la que se almacenan los códigos y las claves, pero que es imposible leer desde cualquier medio externo.
%%
%%En el caso de que la batería del transmisor llega a una tensión peligrosamente baja, se transmite un bit más, avisando así que necesita sustitución o recarga. En este caso se genera un código de transmisión de 67 bits en lugar de los 66 habituales.
%%
%%\section{Cómo funciona el transmisor}
%%
%%El valor de un contador de sincronización de 16 bits se incrementa en cada transmisión al pulsar cualquier botón.
%%
%%Este valor del contador se combina con la clave por medio del algoritmo de cifrado no lineal KeeLoq. Se trata de un complejo algoritmo hace que el cambio de un solo bit en uno de los parámetros de entrada genere un gran cambio en los resultados de salida.
%%
%%El número de serie es único en cada transmisor y lo identifica como un sistema conocido.
%%
%%El estado de los botones indica qué botón o combinación de botones se ha pulsado.
%%
%%\begin{wrapfigure}{r}{9cm}
%%\includegraphics[scale=0.4]{imagenes/funcionamiento_transmisor_keeloq}
%%\caption{\label{fig: Funcionamiento del transmisor keeloq}Funcionamiento del transmisor keeloq.}
%%\end{wrapfigure}
%%
%%El código de control lo forman 2 bits que permiten identificar si la secuencia completa reciba es correcta. Usa un algoritmo CRC (Código de Redundancia Cíclica, muy usado en comunicaciones) Si este sistema identifica que la secuencia recibida no es correcta a causa de ruidos o interferencias, el decodificador la rechaza. Además puede llevar un bit más de identificación de batería baja.\\
%%
%%\section{Cómo funciona el receptor}
%%
%%El decodificador recibe unos datos válidos, el CRC los identifica como datos correctos.
%%
%%\begin{wrapfigure}{r}{9cm}
%%\includegraphics[scale=0.5]{imagenes/funcionamiento_receptor_keeloq}
%%\caption{\label{fig: Funcionamiento del receptor keeloq}Funcionamiento del receptor keeloq.}
%%\end{wrapfigure}
%%
%%El número de serie del transmisor coincide con uno conocido.
%%
%%Combinando los datos cifrados recibidos con la clave almacenada se deduce el valor del contador de sincronización.
%%
%%Si el valor del contador cae dentro del rango válido para ese transmisor, aceptará los comandos establecidos por la combinación de botones.
%%
%%El \textit{HCS500} dispone de siete ranuras de memoria para almacenar los códigos de siete transmisores diferentes.
%%
%%\section{El contador de sincronización}
%%
%%El valor de sincronización es un contador binario de 16 bits. Con 16 bits (216) permite 65536 valores diferentes. El decodificador guarda el valor actual del contador de sincronización de cada transmisor.
%%
%%\begin{wrapfigure}{o}{8cm}
%%\includegraphics[scale=0.6]{imagenes/contador_sincronizacion_keeloq}
%%\caption{\label{fig: Esquema del contador de sincronización}Esquema del contador de sincronización.}
%%\end{wrapfigure}
%%
%%Si el valor recibido para un determinado transmisor coincide con el almacenado en el decodificador o con alguno de los 15 siguientes, reconoce el comando como válido.
%%
%%Si el valor recibido no está en la ventana de operación simple, pero sí en la ventana de operación doble, entonces se almacena el valor recibido en un registro temporal y espera otra transmisión. Cuando recibe un nuevo código del mismo transmisor, se compara con el valor almacenado en el registro temporal. Si los dos valores son consecutivos, aunque el transmisor se ha escapado de la ventana simple, vuelve a estar sincronizado, entonces almacena el nuevo valor del contador de sincronismo y ejecuta el comando.
%%
%%Si un transmisor se ha escapado de la ventana de operación doble ya no funcionará. Será preciso que el decodificador vuelva a reconocerlo en modo aprendizaje. Todas las ventanas rotan cada vez que se recibe una transmisión válida. Esto hace que los valores ya usados estén en la ventana bloqueada y ya no sean válidos, protegiendo así contra el robo de códigos.
%%
%%\section{Cómo se genera la clave}
%%
%%Para que funcione este sistema, el decodificador debe conocer los valores de cada transmisor y tenerlos almacenados y actualizados en su memoria. Existe, por tanto, un proceso de aprendizaje por parte del decodificador. Para esto hay que poner el decodificador en modo aprendizaje. El decodificador debe aprender, por cada transmisor, los siguientes datos:
%%
%%\begin{enumerate}
	%%\item El número de serie único de cada transmisor.
	%%\item El valor del contador de sincronización.
  %%\item La clave de encriptación.	
	%%\item El código del fabricante.
%%\end{enumerate}
%%
%%\section{Aprendizaje del decodificador}
%%
%%Hay varios métodos para que el receptor aprenda los valores de cada transmisor. Cada método tiene su nivel de seguridad y una facilidad diferente que depende del sistema usado para generar la clave del transmisor.
%%
%%\subsection{Método normal}
%%
%%Derivado del número de serie. Se pone el receptor en modo aprendizaje. Al recibir una transmisión normal deduce la clave partiendo de:
%%
%%\begin{enumerate}
	%%\item Número de serie que recibe del transmisor.
	%%\item Clave única de cada fabricante, que debe conocer puesto que la lleva grabada.
	%%\item Algoritmo KeeLoq que lleva programado.
%%\end{enumerate}
%%
%%La relación entre el número de serie y la clave es muy compleja, no evidente para extraños. El decodificador guarda esta clave después de comprobar que en una segunda transmisión decodifica un número de secuencia consecutivo.
%%
%%El aprendizaje es:
%%
%%\begin{itemize}
	%%\item sencillo, sólo requiere poner el receptor en modo aprendizaje y activar dos veces el transmisor.
	%%\item seguro mientras se mantengan secretos:
	%%\begin{itemize}
		%%\item el algoritmo o proceso KeeLoq.
		%%\item la clave del fabricante, única para todos sus sistemas.
	%%\end{itemize}	
%%\end{itemize}
%%
%%\subsection{Método seguro}
%%
%%Derivado de una semilla. No basa su seguridad ni en el secreto del algoritmo, ni en el secreto de la clave del fabricante. Para un extraño que conozca el algoritmo, las transmisiones seguirán siendo incomprensibles si no conoce la clave del transmisor.
%%
%%Según Microchip, determinar la clave analizando las transmisiones no es factible. Se necesitaría un ordenador muy potente y mucho tiempo de proceso para deducir la clave de un solo transmisor. Claro que esto es cada vez menos cierto dado el avance de la informática.
%%
%%En el momento de su fabricación se programa en cada transmisor:
%%
%%un número de serie una semilla una clave propia
%%la clave del fabricante
%%No hay relación entre el número de serie y la clave. Hay una relación muy compleja, pero fija, entre la semilla y la clave. Durante el aprendizaje del receptor, el transmisor envía la semilla al efectuar en él una operación especial. Nunca se transmite la semilla en una operación normal del transmisor. Si se desea, se puede programar al transmisor para que, después de un determinado número de operaciones, jamás pueda volver a transmitir la semilla.
%%
%%Cuando el decodificador, puesto en modo aprendizaje, recibe la semilla desde el transmisor, deduce la clave usando el algoritmo KeeLoq. Como nunca mas se transmitirá la semilla, no hay peligro de que nadie deduzca nunca la clave.
%%
%%\section{Glosario de términos}
%%
%%\textbf{Número de serie del transmisor}
%%
%%Cada transmisor tiene un número de serie único de 28 ó 32 bits.\\
%%
%%\textbf{Clave secreta}
%%
%%Clave de 64 bits generada por una función partiendo de los siguientes valores:
%%
%%\begin{itemize}
	%%\item Número de serie de 28 ó 32 bits o bien por una semilla de 32 ó 48 bits.
	%%\item Clave del fabricante, de 64 bits.
%%\end{itemize}
%%
%%Jamás se transmite y no es posible leerla.\\
%%
%%\textbf{Semilla}
%%
%%Valor de 32 ó 48 bits programado en el codificador. Se transmite al pulsar una combinación concreta de botones. Se puede deshabilitar después del aprendizaje del decodificador.\\
%%
%%\textbf{Generador de la clave}
%%
%%Función matemática usada para generar una clave única para cada transmisor partiendo del número de serie o de la semilla.\\
%%
%%\textbf{Clave del fabricante}
%%
%%Es necesaria en el receptor si se ha usado en la función de generación de la clave. Puede ser programada en el receptor en el proceso de fabricación.\\
%%
%%\textbf{Aprendizaje normal (derivado del número de serie)}
%%
%%El receptor usa la misma información usada en una transmisión normal para generar la clave secreta y descifrar el valor del número de secuencia del transmisor.\\
%%
%%\textbf{Aprendizaje seguro (derivado de la semilla)}
%%
%%El transmisor se activa con una combinación especial de botones para emitir un valor de 32 ó 48 bits (semilla) que se usa para generar la clave o ser parte de la clave. La transmisión de la semilla puede ser deshabilitada después del aprendizaje.\\
%%
%%\textbf{Contador de sincronización}
%%
%%Contador de 16 bits que se incrementa en cada activación del codificador.
%%
%%
%%
%%
